\documentclass[spanish]{article}

\usepackage{mypack}

\begin{document}

\begin{small}\noindent
  \vspace{.5cm}

    \begin{theorem}
    [Ecuaciones de la recta] Recta que pasa por el punto $P\left(p_x,p_y,p_z\right)$ y cuyo vector director es $\vec{u}_r=\left(u_{rx},u_{ry},u_{rz}\right)$:
    \begin{itemize}
    \item Ecuaci�n vectorial: $\overrightarrow{OX}=\overrightarrow{OP}+\lambda\vec{u}_r$, $(x,y,z)=(p_x,p_y,p_z)+\lambda(u_{rx},u_{ry},u_{rz})$.
    \item Ecuaciones param�tricas:$\left\{\begin{array}{l}x=p_x+\lambda u_{rx} \\ y=p_y+\lambda u_{ry} \\ z=p_z+\lambda u_{rz}\end{array}\right.$
    \item Ecuaci�n continua: $\dfrac{x-p_x}{u_{rx}}=\dfrac{y-p_y}{u_{ry}}=\dfrac{z-p_z}{u_{rz}}$.
    \end{itemize}
  \end{theorem}
\vspace{.5cm}

\end{small}

\begin{enumerate}
\item Escribe las ecuaciones (vectorial, param�tricas y continua) de la recta que pasa por el punto $P(5,0,-1)$ y tiene la direcci�n de $\vec{u}=(3,1,-2)$.
  \vspace{3cm}
  
\item Halla dos puntos y un vector director de cada una de las rectas siguientes:
  \begin{enumeratealpha}
    \item $r_1:(x,y,z)=(5,-2,-2)+\lambda(2,1,0)$
  \vspace{3cm}
  
    \item $r_2:\left\{\begin{array}{l}x=3-\lambda\\y=5\\z=-1+4\lambda\end{array}\right.$
  \vspace{3cm}
  
    \item $r_3\dfrac{x+1}{3}=\dfrac{y}{5}=z-2$
  \vspace{3cm}
  
    \item $r_4\dfrac{x-3}{2}=\dfrac{y+3}{-1}=\dfrac{2z}{3}$
  \vspace{3cm}
  
  \end{enumeratealpha}
\item Escribe las ecuaciones param�tricas de cada una de las rectas siguientes:
  \begin{enumeratealpha}
    \item $r: x=y=z$
  \vspace{3cm}
  
    \item $s\equiv\dfrac{x}{2}=\dfrac{y+3}{5}=\dfrac{z}{-3}$
  \vspace{3cm}
  
  \end{enumeratealpha}
\item Di si las rectas siguientes pasan por el punto que se indica en cada caso:
  \begin{enumeratealpha}
    \item $r_1\equiv\left\{\begin{array}{l}x=1+2\lambda\\y=2-\lambda\\z=5-3\lambda\end{array}\right.$, �pasa por $P(5,0,-1)$?
  \vspace{1cm}
  
    \item $r_2\equiv\dfrac{x+3}{2}=\dfrac{y-1}{5}=\dfrac{z+3}{-2}$, �pasa por $P(3,16,5)$?
  \vspace{1cm}
  
    \item $r_3\equiv(x,y,z)=(1,1,2)+\lambda(-3,0,2)$, �pasa por $P(4,1,0)$?
  \vspace{1cm}
  
  \end{enumeratealpha}
\item Escribe las ecuaciones (vectorial, param�tricas y continua) de una recta que pase por $P(2,1,-1)$ y sea paralela a la recta $r:\dfrac{x-2}{3}=\dfrac{y-1}{2}=z$. Nota: dos rectas son paralelas si tienen la misma direcci�n (es decir, el mismo vector director).
  \vspace{3cm}
  
\item Escribe las ecuaciones (vectorial, param�tricas y continua) de una recta que pase por $P(2,1,-1)$ y sea ortogonal a la recta $r:\left\{\begin{array}{l}x=4\\y=1+\lambda\\z=-1+4\lambda\end{array}\right.$. Nota: decimos que dos rectas son ortogonales o perpendiculares cuando sus vectores directores son ortogonales. Por lo tanto, no es necesario que las rectas se corten en un punto. Este ejercicio tiene infinitas soluciones.



\end{enumerate}
\end{document}
