\documentclass[spanish,svgnames]{article}

\usepackage{mypack}
%\usepackage[svgnames]{xcolor}

% \usepackage{PTSerif} % Use the Paratype Serif font


\graphicspath{{./img/}}

\begin{document}
\noindent {\large\textbf{Ejercicios 10-02-21}}
\vspace{1cm}

\noindent\textbf{249}. Demostrar que la ecuaci�n $x^3+x^2-7x+1=0$ tiene al menos una soluci�n en el intervalo $[0,1]$.

\noindent\textbf{251 (c,h,j,l)}.
  Calcula el dominio y las as�ntotas de las funciones siguientes:
  \begin{enumeratealpha}
  \setcounter{enumi}{2}
    \item $f(x)=\dfrac{x^2-x-2}{x^2-4x+4}$
  \setcounter{enumi}{7}
    \item $f(x)=\dfrac{3x^2-5}{x+1}$
  \setcounter{enumi}{9}
    \item $f(x)=\sqrt{\dfrac{x^2-4}{x^2-1}}$
  \setcounter{enumi}{11}
    \item $f(x)=\dfrac{2x-3}{|x-1|}$
\end{enumeratealpha}

\noindent\textbf{252}.   Determinar el valor de la constante $k$ sabiendo que la curva de ecuaci�n $y=\dfrac{x^3+kx^2+1}{x^2+1}$ posee una as�ntota que pasa por el punto $(1,3)$.


\end{document}
