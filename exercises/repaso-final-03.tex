\documentclass[spanish,11pt]{article}

\usepackage{mypack}

\begin{document}

\noindent \fbox{\begin{minipage}{17.6cm}
    \noindent \textbf{Matem�ticas II. Repaso Geometr�a Anal�tica.}

    \noindent \textbf{Prueba de velocidad: �reas, vol�menes, distancias y �ngulos.}
\end{minipage}}

\vspace{.5cm}

\noindent Resuelve el mayor n�mero posible de ejercicios. En cada uno escribe solo la soluci�n.

\vspace{.5cm}

\noindent \begin{tabular}{ | p{.3cm} | p{12cm} | p{4.5cm} | } \hline
            N� & Problema & Soluci�n \\ \hline
            1 & Hallar la distancia entre $r:\dfrac{x}{1}=\dfrac{y-1}{-2}=\dfrac{z-3}{2}$ y $s:\dfrac{x-2}{3}=\dfrac{y}{1}=\dfrac{z+1}{-1}$.  \vspace{.3cm} & \rule{5cm}{0pt} \rule[-7mm]{0pt}{1.3cm} \\ \hline
            2 & Determinar el punto sim�trico del punto $M(1,1,1)$ respecto del plano $\pi:x+y-2z=6$. \vspace{.3cm} & \rule{5cm}{0pt} \rule[-7mm]{0pt}{1.3cm} \\ \hline
            3 & Determinar el punto sim�trico del punto $M(1,1,1)$ respecto de la recta $r:\dfrac{x-1}{2}=\dfrac{y}{3}=\dfrac{z+1}{-1}$ \vspace{.3cm} & \rule{5cm}{0pt} \rule[-7mm]{0pt}{1.2cm} \\ \hline 
            4 & Calcular la distancia de la recta $r: \left\{\begin{array}{l}x=2-3\lambda\\y=1+2\lambda\\z=4-\lambda\end{array}\right. $ al plano definido por $\pi:3+2x+2y-2z=0$. \vspace{.3cm} & \rule{5cm}{0pt} \rule[-7mm]{0pt}{1.2cm} \\ \hline 
            5 &  Hallar el volumen del tetraedro cuyos v�rtices son el origen y los puntos de intersecci�n de $\pi:x+2y+3z=6$ con los ejes de coordenadas. \vspace{.3cm} & \rule{5cm}{0pt} \rule[-7mm]{0pt}{1.2cm} \\ \hline 
            6 & Hallar los puntos $A$ contenidos en la recta $r: \left\{\begin{array}{l}x-y=0\\x+2y+3z=0\end{array}\right. $  tales que el tri�ngulo de v�rtices $A$, $P(1,1,1)$ y $O(0,0,0)$ tenga �rea $1$. \vspace{.3cm}& \rule{5cm}{0pt} \rule[-7mm]{0pt}{1.3cm} \\ \hline 
            7 & Calcular el coseno del �ngulo que forman el plano $\pi:x+3y+z=4$ y el plano $z=0$. \vspace{.3cm} & \rule{5cm}{0pt} \rule[-1cm]{0pt}{1.4cm} \\ \hline
            8 & Hallar las ecuaciones de los planos paralelos a $\pi \equiv 2x-y+2z+1-0$, que se encuentran a $3$ unidades de distancia de $\pi$. \vspace{.3cm} & \rule{5cm}{0pt} \rule[-1.1cm]{0pt}{1.7cm} \\ \hline 
            9 & Hallar el punto de la recta $r\equiv \dfrac{x-2}{3}=\dfrac{y}{-1}=\dfrac{z-4}{2}$ que equidistan de $A(2,2,3)$ y $B(0,-2,1)$. \vspace{.3cm} & \rule{5cm}{0pt} \rule[-7mm]{0pt}{1.2cm} \\ \hline 
            10 & Hallar la distancia del punto $A(0,1,-1)$ a la recta $s\equiv \left\{\begin{array}{l}x+z=3\\2x-y=2\end{array}\right. $. \vspace{.3cm} & \rule{5cm}{0pt} \rule[-7mm]{0pt}{1.2cm} \\ \hline 
\end{tabular}



\end{document}