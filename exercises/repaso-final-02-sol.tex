\documentclass[spanish,11pt]{article}

\usepackage{mypack}

\begin{document}

\noindent \fbox{\begin{minipage}{17.6cm}
    \noindent \textbf{Matem�ticas II. Repaso Geometr�a Anal�tica.}

    \noindent \textbf{Prueba de velocidad: obtenci�n de rectas y planos.}
\end{minipage}}

\vspace{.5cm}

\noindent Resuelve el mayor n�mero posible de ejercicios. En cada uno escribe solo la soluci�n. Escribe los planos en forma general, y las rectas en forma continua o en param�tricas.

\vspace{.5cm}

\noindent \begin{tabular}{ | p{.3cm} | p{12cm} | p{5cm} | } \hline
            N� & Problema & Soluci�n \\ \hline
            1 & Hallar el plano paralelo a $\alpha:2x+y+2z+1$ que pasa por el punto $P(2,1,0)$.  \vspace{.3cm} & $2x+y+2z-5=0$\rule[-7mm]{0pt}{1.3cm} \\ \hline
            2 & Hallar el plano determinado por las rectas $r\equiv x=\dfrac{y-1}{2}=\dfrac{z+1}{-1}$ y $s \equiv \left\{\begin{array}{l}x+z=3\\2x-y=2\end{array}\right. $. \vspace{.3cm} & $5x-4y-3z+1=0$\rule[-7mm]{0pt}{1.3cm} \\ \hline
            3 & Hallar el plano que pasa por el punto $A(-1,-2,-3)$, es paralelo a la recta $r: \left\{\begin{array}{l}x+y+1=0\\z=0\end{array}\right. $ y perpendicular a $\pi:x-2y-3z+100$.\vspace{.3cm} & $3x+3y-z+6=0$\rule[-7mm]{0pt}{1.2cm} \\ \hline 
            4 & Obtener el plano que contiene a la recta $r: \left\{\begin{array}{l}x-2z-1=0\\x+y+z-4=0\end{array}\right. $ y es paralelo a $s:\{(2+\lambda,1-3\lambda,\lambda);\lambda\in\mathbb{R}\}$.\vspace{.3cm} & $y+3z=3$\rule[-7mm]{0pt}{1.2cm} \\ \hline 
            5 & Hallar la ecuaci�n del plano que contiene a la recta $r: \left\{\begin{array}{l}x=1+t\\y=-1+2t\\z=t\end{array}\right. $ y es perpendicular al plano $\pi:2x+y-z=2$.\vspace{.3cm} & $x-y+z-2=0$\rule[-7mm]{0pt}{1.2cm} \\ \hline 
            6 & Hallar la recta que pasa por $D(2,-1,-2)$ y es perpendicular al plano determinado por los puntos $A(1,1,1)$, $B(0,-2,2)$ y $C(-1,0,2)$. \vspace{.3cm}& $\dfrac{x-2}{2}=\dfrac{y+1}{1}=\dfrac{z+2}{5}$\rule[-7mm]{0pt}{1.3cm} \\ \hline 
            7 & Calcular la recta contenida en $\pi:x+2y-z=2$ que pasa por el punto $P(-2,3,2)$ y que corta perpendicularmente a $r:\dfrac{x-3}{2}=\dfrac{y-2}{1}=\dfrac{z-5}{4}$.\vspace{.3cm} & $\dfrac{x+2}{3}=\dfrac{y-3}{-2}=\dfrac{z-2}{-1}$\rule[-1cm]{0pt}{1.4cm} \\ \hline 
            8 & Determinar la perpendicular com�n a las rectas $r: \left\{\begin{array}{l}x-z=1\\y-z=-1\end{array}\right. $ y $s: \left\{\begin{array}{l}x=1+\lambda\\y=\lambda\\z=3\end{array}\right. $.\vspace{.3cm} & $ \left\{\begin{array}{l}x=6-\lambda\\y=\lambda\\z=3\end{array}\right. $\rule[-1.1cm]{0pt}{1.7cm} \\ \hline 
            9 & Hallar la recta que pasa por el origen de coordenadas y es perpendicular a la recta $r: \left\{\begin{array}{l}x-y=3\\x+y-z=0\end{array}\right. $ y a la recta $s: \left\{\begin{array}{l}x-z=4\\2x-y=7\end{array}\right. $.\vspace{.3cm} & $\dfrac{x}{3}=-y=-z$\rule[-7mm]{0pt}{1.2cm} \\ \hline 
            10 & Calcular la recta que pasa por $P(3,-1,0)$ y corta perpendicularmente a $r: \left\{\begin{array}{l}x=3+2\lambda\\y=4+\lambda\\z=5+3\lambda\end{array}\right. $\vspace{.3cm} & $\dfrac{x-3}{4}=\dfrac{y+1}{-5}=\dfrac{z}{-1}$\rule[-7mm]{0pt}{1.2cm} \\ \hline 
\end{tabular}



\end{document}