\documentclass[spanish,11pt]{article}

\usepackage{mypack}

\begin{document}

\noindent \textbf{Pr�ctica de Integrales para Semana Santa} \hfill 19 de marzo de 2018
\vspace{.5cm}

\noindent\fbox{\noindent\textbf{Haz DOS ejercicios cada d�a: 10-20 minutos por ejercicio = 20-40 minutos.}}

\begin{nexercise}
  Calcula $\displaystyle\int \! \dfrac{1}{(x-1)^2} \, \mathrm{d}x$
\end{nexercise}


\begin{nexercise}
  Calcula: $\displaystyle\int_{-1}^2 \! x|x| \, \mathrm{d}x$.
\end{nexercise}

\begin{nexercise}
  Calcula: $\displaystyle\int \! \dfrac{x+3}{\sqrt{9-x^2}} \, \mathrm{d}x$
\end{nexercise}

\begin{nexercise}
  Calcula: $\displaystyle\int_{\pi/4}^{\pi/2} \! x\sen x \, \mathrm{d}x$.
\end{nexercise}

\begin{nexercise}
  Calcula los valores de $a$ para los que: $\displaystyle\int_1^a \! \dfrac{8}{x^3} \, \mathrm{d}x = 3$.
\end{nexercise}

\begin{nexercise}
  Calcula $\displaystyle\int_0^{\sqrt{3}} \! \dfrac{2x^3}{\sqrt{x^2+1}} \, \mathrm{d}x$, mediante la sustituci�n $t=\sqrt{x^2+1}$.
\end{nexercise}

\begin{nexercise}
  Calcula: $\displaystyle\int_0^{\pi/4} \! \sen x \cdot \cos x \, \mathrm{d}x$.
\end{nexercise}

\begin{nexercise}
  Calcula $\displaystyle\int \! \dfrac{\cos x}{\sen^3 x} \, \mathrm{d}x$
\end{nexercise}

\begin{nexercise}
  Calcula el �rea de la regi�n delimitada por las curvas $y=x^2$ e $y=4x-x^2$.
\end{nexercise}

\begin{nexercise}
  Encuentra la primitiva de $f(x) = (x+2)\ln x$, para $x>0$, que pasa por el punto $(1,0)$.
\end{nexercise}

\begin{nexercise}
  Encuentra la primitiva de $f(x)=\dfrac{x^2}{\left(1+x^3\right)^{3/2}}$ que pasa por el punto $(2,0)$. Utiliza la sustituci�n $t=1+x^3$.
\end{nexercise}

\begin{nexercise}
  Halla la ecuaci�n de la recta tangente a la gr�fica de una funci�n $f$ en el punto de abscisa $x=1$ sabiendo que $f(0)=0$ y $f'(x)=\dfrac{(x-1)^2}{x+1}$ para $x>-1$.
\end{nexercise}

\begin{nexercise}
  Calcula el �rea de la regi�n del plano limitada por la curva $y=\dfrac{x}{x^2+5x+4}$, el eje OX y las rectas $x=0$, $x=2$.
\end{nexercise}

\begin{nexercise}
  Calcula el �rea de la regi�n delimitada por la gr�fica de $f(x)=\dfrac{\ln x}{x}$ y las rectas $y=0$, $x=1$, $x=e$.
\end{nexercise}

\begin{nexercise}
  Calcula $\displaystyle\int_0^8 \! \dfrac{1}{2+\sqrt{x+1}} \, \mathrm{d}x$ utilizando el cambio de variable $t=2+\sqrt{x+1}$.
\end{nexercise}

\begin{nexercise}
  Calcula el �rea de la regi�n limitada por la gr�fica de $f(x)=\sqrt{2x-2}$, la recta $y=x-5$ y el eje de abscisas.
\end{nexercise}

\begin{nexercise}
  Calcula $\displaystyle\int_3^3 \! \sqrt{3+|x|} \, \mathrm{d}x$.
\end{nexercise}

\begin{nexercise}
  Dada la funci�n $f(x)=ax^3+bx^2+c$, hallar $a$, $b$ y $c$ para que la funci�n tenga un m�nimo en el punto $(2,-3)$, y verifique $\displaystyle\int_0^2 \! f(x) \, \mathrm{d}x = -2$.
\end{nexercise}

\begin{nexercise}
  Halla una primitiva de $f(x)=\dfrac{x}{1+x^4}$ que verifique $f(1)=0$.
\end{nexercise}

\begin{nexercise}
  Calcula $\displaystyle\int_0^x \! t^2e^{-t} \, \mathrm{d}x$.
\end{nexercise}

\newpage

\noindent \textbf{Soluciones:}

\setcounter{exercisenr}{0}

\begin{exercise}
  $I = \dfrac{-1}{x-1}+C$
\end{exercise}

\begin{exercise}
  $A=\dfrac{7}{3}$
\end{exercise}

\begin{exercise}
  $I = -\sqrt{9-x^2} + 3\arcsen \left(\dfrac{x}{3}\right) + C$
\end{exercise}

\begin{exercise}
  $I=\left[\sen(x)-x\cos(x)\right]_{\pi/4}^{\pi/2} = 0.8483$
\end{exercise}

\begin{exercise}
  $a=\pm2$
\end{exercise}

\begin{exercise}
  $I=\dfrac{8}{3}$
\end{exercise}

\begin{exercise}
  $I=\left[\dfrac{\sen^2x}{2}\right]_0^{\pi/4} = \dfrac{1}{4}$
\end{exercise}

\begin{exercise}
  $I = -\dfrac{1}{2\sen^2x} + C$
\end{exercise}

\begin{exercise}
  $A=\displaystyle\int_0^2 \! (4x-x^2-x^2) \, \mathrm{d}x = \dfrac{8}{3}$
\end{exercise}

\begin{exercise}
  $F(x) = \left(\dfrac{x^2}{2}+2x\right)\ln x - \dfrac{1}{4}x^2 - 2x + \dfrac{9}{4}$.
\end{exercise}

\begin{exercise}
  $-\dfrac{2}{3\sqrt{1+x^3}} + \dfrac{2}{9}$
\end{exercise}

\begin{exercise}
  $y=-\dfrac{5}{2}+4\ln2$
\end{exercise}

\begin{exercise}
  $A=\left[\dfrac{4}{3}\log|x+4|-\dfrac{1}{3}\log|x+1|\right]_0^2 = \dfrac{4}{3}\ln6-\dfrac{4}{3}\ln4-\dfrac{1}{3}\ln3 = 0.1744$
\end{exercise}

\begin{exercise}
  $A=\left[\dfrac{1}{2}\ln^2x\right]_1^e = \dfrac{1}{2}$
\end{exercise}

\begin{exercise}
  $I = 2\left[t-2\ln t\right]_3^5 = 4 + 4\ln \dfrac{3}{5}$.
\end{exercise}

\begin{exercise}
  $A=\dfrac{40}{3}$
\end{exercise}

\begin{exercise}
  $I=8\sqrt{6}-4\sqrt{3}$
\end{exercise}

\begin{exercise}
  $a=1;\ b=-3;\ c=1$
\end{exercise}

\begin{exercise}
  $F(x) = \dfrac{1}{2}\arctg\left(x^2\right)-\dfrac{\pi}{8}$
\end{exercise}

\begin{exercise}
  $F(x)=-e^{-x}(x^2+2x+2)+2$
\end{exercise}

\end{document}